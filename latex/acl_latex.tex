% This must be in the first 5 lines to tell arXiv to use pdfLaTeX, which is strongly recommended.
\pdfoutput=1
% In particular, the hyperref package requires pdfLaTeX in order to break URLs across lines.

\documentclass[11pt]{article}

% Change "review" to "final" to generate the final (sometimes called camera-ready) version.
% Change to "preprint" to generate a non-anonymous version with page numbers.
\usepackage[review]{acl}

% Standard package includes
\usepackage{times}
\usepackage{latexsym}

% For proper rendering and hyphenation of words containing Latin characters (including in bib files)
\usepackage[T1]{fontenc}
% For Vietnamese characters
% \usepackage[T5]{fontenc}
% See https://www.latex-project.org/help/documentation/encguide.pdf for other character sets

% This assumes your files are encoded as UTF8
\usepackage[utf8]{inputenc}

% This is not strictly necessary, and may be commented out,
% but it will improve the layout of the manuscript,
% and will typically save some space.
\usepackage{microtype}

% This is also not strictly necessary, and may be commented out.
% However, it will improve the aesthetics of text in
% the typewriter font.
\usepackage{inconsolata}

%Including images in your LaTeX document requires adding
%additional package(s)
\usepackage{graphicx}

% If the title and author information does not fit in the area allocated, uncomment the following
%
%\setlength\titlebox{<dim>}
%
% and set <dim> to something 5cm or larger.

\title{Instructions for *ACL Proceedings}

% Author information can be set in various styles:
% For several authors from the same institution:
% \author{Author 1 \and ... \and Author n \\
%         Address line \\ ... \\ Address line}
% if the names do not fit well on one line use
%         Author 1 \\ {\bf Author 2} \\ ... \\ {\bf Author n} \\
% For authors from different institutions:
% \author{Author 1 \\ Address line \\  ... \\ Address line
%         \And  ... \And
%         Author n \\ Address line \\ ... \\ Address line}
% To start a separate ``row'' of authors use \AND, as in
% \author{Author 1 \\ Address line \\  ... \\ Address line
%         \AND
%         Author 2 \\ Address line \\ ... \\ Address line \And
%         Author 3 \\ Address line \\ ... \\ Address line}

\author{First Author \\
  Affiliation / Address line 1 \\
  Affiliation / Address line 2 \\
  Affiliation / Address line 3 \\
  \texttt{email@domain} \\\And
  Second Author \\
  Affiliation / Address line 1 \\
  Affiliation / Address line 2 \\
  Affiliation / Address line 3 \\
  \texttt{email@domain} \\}

%\author{
%  \textbf{First Author\textsuperscript{1}},
%  \textbf{Second Author\textsuperscript{1,2}},
%  \textbf{Third T. Author\textsuperscript{1}},
%  \textbf{Fourth Author\textsuperscript{1}},
%\\
%  \textbf{Fifth Author\textsuperscript{1,2}},
%  \textbf{Sixth Author\textsuperscript{1}},
%  \textbf{Seventh Author\textsuperscript{1}},
%  \textbf{Eighth Author \textsuperscript{1,2,3,4}},
%\\
%  \textbf{Ninth Author\textsuperscript{1}},
%  \textbf{Tenth Author\textsuperscript{1}},
%  \textbf{Eleventh E. Author\textsuperscript{1,2,3,4,5}},
%  \textbf{Twelfth Author\textsuperscript{1}},
%\\
%  \textbf{Thirteenth Author\textsuperscript{3}},
%  \textbf{Fourteenth F. Author\textsuperscript{2,4}},
%  \textbf{Fifteenth Author\textsuperscript{1}},
%  \textbf{Sixteenth Author\textsuperscript{1}},
%\\
%  \textbf{Seventeenth S. Author\textsuperscript{4,5}},
%  \textbf{Eighteenth Author\textsuperscript{3,4}},
%  \textbf{Nineteenth N. Author\textsuperscript{2,5}},
%  \textbf{Twentieth Author\textsuperscript{1}}
%\\
%\\
%  \textsuperscript{1}Affiliation 1,
%  \textsuperscript{2}Affiliation 2,
%  \textsuperscript{3}Affiliation 3,
%  \textsuperscript{4}Affiliation 4,
%  \textsuperscript{5}Affiliation 5
%\\
%  \small{
%    \textbf{Correspondence:} \href{mailto:email@domain}{email@domain}
%  }
%}

\begin{document}
\maketitle
\begin{abstract}
.....
\end{abstract}

\section{Introduction}


\begin{itemize}
    \item The importance of plasticity in deep learning.
    \item Application of that in real world , etc 
    \item Overall categorization of what the literature did : their questions and solutions 
    \item What is the gap here
    \item what we did
    \item Our contribution
\end{itemize}



\section{Related Work}

We

\section{What is plasticity and how to measure that}

~ Half page write-up
\begin{itemize}
    \item Explain the literature and their evaluation metrics (as Eq?)
    \item Explain what we would consider
    \item Methods to reduce LoP - l2 reg, layer norm, crelu etc.
\end{itemize}


Example exp:
\begin{itemize}
    \item the following CIFAR10 exp shows loss of plasticity when ... 
    \item Show lop in terms of evaluation metric 
    \item Show how methods like L2, layer norm help reduce LoP.
\end{itemize}


% \subsection{Plasticity in NLP}

% \subsection{Plasticity in self-attention networks}

\section{Experiments}

\subsection{Setup}
List all:
\begin{itemize}
    \item What is the datasets,
    \item  models, 
    \item CL setup: number of tasks, 
    \item optimizers, hyper-param setup (details about the grid search over lr). 
    \item Random label ( 100\%, 50\%, 20\%)
\end{itemize}
Also, mention why did you choose this particular setup.


\subsection{Results with Transformer (default setup)}

pretrain model, 

\subsubsection{Effect of pre-training - pre trained bert}
\begin{itemize}
    \item Bert Model (pre-trained and no-pretrained)
    \item is that because of pretrained and non-pretrained
    \item is that because of transformers?
    % \item is that because of scale?
\end{itemize}
 % --- is that because of pretrained and non-pretrained 

 % --- is that because of transformers

\subsection{Ablations/Intervention/Analysis}

\subsubsection{Experiments for other type of networks: CNN, LSTM}
\begin{itemize}
    \item CNN on NLP results. Is there plasticity loss?
    \item LSTM on NLP results. Is there plasticity loss?
\end{itemize}

\subsection{Comparison with vision}

??? 


\subsubsection{self-attention vs MLP}
\begin{itemize}
    \item Self Attention Layer (Embedding + Multi-head Attention layer + Classifier)
    \item Self Attention Layer  with Residual Layer
    \item Bert Layer( Layer Norm and dropout)
\end{itemize}





\section{Discussion and Future Works}


\section{Conclusion}

% Bibliography entries for the entire Anthology, followed by custom entries
%\bibliography{anthology,custom}
% Custom bibliography entries only
\bibliography{custom}

\appendix

\section{Example Appendix}
\label{sec:appendix}

This is an appendix.

\end{document}
